%\begin{appendix} %Anhang


%Anhang A
%\includepdf[pages={1},nup=1x1,landscape=false,scale=0.90, pagecommand = \section*{\LARGE{Anhang:}}
%\addcontentsline{toc}{section}{Anhang}
%\section{Auftrag des Arbeitgebers}
%\label{app:Lastenheft}, offset =0mm -22mm ]{appendix/Lastenheft.pdf}\newpage
%Bei mehrseitigen Dokumenten die folgenden Seiten ohne Überschrift:
%\includepdf[pages={2},nup=1x1,landscape=false,scale=0.90,offset=6 -30,pagecommand={\thispagestyle{myheadings}}]{appendix/Lastenheft.pdf} 
%\newpage


%Anhang B: Bestimmung der Ersatzelemente der ASM
%\input{sections/ASM_Laborjournal}


%Anhang C: Messungen für ADC-Verifikation
\clearpage
\section{Messungen Zeitverzögerung Modulator}\label{app:Messungen_Delay Modulator}



\begin{figure}[H]
	\centering
	\includegraphics[width=0.85\linewidth]{appendix/Delay_modulator.png}
	\caption{Messung der Zeitverzögerung zwischen Modulatoreingang und modulierter Spannung an der ASM}
	\label{fig:app_delay}
\end{figure}


%Beispiel für A3 Seite einfügen:

%Anhang B
%A3
%\eject\pdfpagewidth=420mm \pdfpageheight=297mm
%A3


%Anhang B: Schema Layout
%\includepdf[pages={1},nup=1x1,landscape=false,scale=1.7,offset=300 -50,pagecommand={\section{Schema BMS-Slave}\label{app: Schema_Layout}\thispagestyle{myheadings}}]{appendix/Schema_Layout_.pdf} \newpage

%A4
%\eject\pdfpagewidth=210mm \pdfpageheight=297mm
%A4


%Beispiel Anhang C: (PDF)



%Anhang C: Dimensionierung passives Balancing
%\includepdf[pages={1},nup=1x1,landscape=false,scale=0.90,offset=10 -40,pagecommand={\section{Dimensionierung passives Balancing}\label{app: Dimensionierung_Balancing}\thispagestyle{myheadings}}]{appendix/Dimensionierung_Balancing.pdf} 
%\newpage


%\end{appendix}
