\clearpage
\section{Dokumentation der Regelauslegung}\label{sec: Regelauslegung}
\vspace{0.5cm}
\textbf{Durchgeführt von:} Fabian von Büren \& Severin Weibel

\textbf{Datum:} \today
\vspace{1cm}

Diese Dokumentation dient dazu die Grenzen unseres Systems festzuhalten. Die Zwischenkreisspannung und die Maschine selber bestimmen die wichtigsten Grössen. Die bestimmten Grenzen dienen in erster Linie der Auslegung vom Flussverlauf und vom Lastverlauf im Feldschwächungsbereich. Die Maschinenparameter stammen aus dem Laborversuch und vom Typenschild. 


\subsection{Maschinenparameter}
Folgende Maschine wird verwendet:

\renewcommand{\arraystretch}{1.5}
\begin{table}[h]
	\centering
	\begin{tabular}{|l|c|}
		\hline
		Hersteller & Siemens \\ \hline
		Maschinentyp & Käfigläufer ASM \\ \hline
		Maschinennummer & 1LE1001-1CB03-4AA4 \\ \hline
		Nennspannung $U_{N}$ (50Hz, 60Hz) & $\Delta 400\mathrm{V}, \Delta 460\mathrm{V}$ \\ \hline
		Nennstrom $I_{N}$ (50Hz, 60Hz) & $11.3\mathrm{A}, 9.9\mathrm{A}$ \\ \hline 
		Nennleistung $P_{me}$ & $5.5\mathrm{kW}$ \\ \hline 
		cos$\varphi$ & 0.8 \\ \hline
		Nenndrehzahl $\omega_{me}$(50Hz) & 1465U/min \\ \hline
		Polpaarzahl & 2 \\ \hline
	\end{tabular} 	
	\caption{Nenndaten für die auszumessende ASM-Maschine}
	\label{tab:Nenndaten_ASM}
\end{table}

Folgende Maschinenparameter werden verwendet (ASM im Labor ausgemessen):
\begin{table}[h]
	\centering
	\begin{tabular}{|c|c|}
		\hline 
		Rotorwiderstand $R_{S}$ & $\quad 2.6/3 \Omega \quad$ \\ \hline 
		Hauptinduktivität $L_{h}$ & $359.9/3 \mathrm{mH}$ \\ \hline
		Streuinduktivität Stator $L_{\sigma,S}$ & $54.23/2/3 \mathrm{mH}$ \\ \hline
		Streuinduktivität Rotor $L_{\sigma,R}$ & $54.23/2/3 \mathrm{mH}$ \\ \hline		
		Statorwiderstand $R_{R}$ & $2.6/3 \Omega$ \\ \hline 	
	\end{tabular} 	
	\caption{Ergebnisse für die untersuchte ASM-Maschine (Faktor 3 aufgrund Stern-Dreieck-Umwandlung)}
	\label{tab:Spezifikation_ASM}
\end{table}
\renewcommand{\arraystretch}{1}


Daraus können wichtige Grössen berechnet werden:

Absorbierte Wirkleistung:
\begin{equation}\label{equ:Absorbierte_Leitung}
P_{N,1} = \sqrt{3}\cdot U_{N}\cdot I_{N}\cdot \cos \varphi = \sqrt{3}\cdot 400\cdot 11.3\cdot 0.8 = 6.263\mathrm{kW}
\end{equation}

Wirkungsgrad der Maschine:
\begin{equation}\label{equ:Wirkungsgrad}
\eta = \frac{P_{me}}{P_{N,1}} = \frac{5.5\cdot 10^{3}}{6.232\cdot 10^{3}} = 87.82\%
\end{equation}

Nennfluss des Rotors:
\begin{equation}\label{equ:Nennfluss}
\Psi_{R,N} = \frac{\hat{U}_{H}}{\omega _{el}} = \frac{\sqrt{2}\cdot(U_n/\sqrt{3}-I_{N}\cdot R_{S})}{2\pi f_{N}} = \frac{\sqrt{2}\cdot(400/\sqrt{3}-11.3\cdot 2.3/3)}{2\pi 50} =  0.995 \mathrm{Vs}
\end{equation}
\todo{Nennfluss ist zu hoch. Simulation wird instabil. Ich rechne mit 90\% davon weiter. Es entstehen ja noch Verluste und somit kann nicht der gesamte Fluss gebraucht werden.}

Nennmoment der Maschine:
\begin{equation}\label{equ:Nennmoment}
M_{N} = \frac{P_{me}}{\omega _{me}} = \frac{5.5\cdot 10^{3}}{2\pi\cdot 1465/60} = 35.86 \mathrm{Nm}
\end{equation}
\todo{Wo finde ich diese Gleichung im Skript?}

Diese Nenngrenzen werden durch das Einsetzen vom einem Stromrichter verändert. Um die genauen Grenzen zu berechnen muss zuerst die Zwischenkreisspannung bestimmt werden. Diese wird wie folgt berechnet:
Von der Literatur ist bekannt, dass die Gleichspannung am Ausgang einer Diodenbrücke bei kapazitiver Glättung sich mit der Formel \ref{equ:Zwischenkreisspannung} berechnet.

\begin{equation}\label{equ:Zwischenkreisspannung}
U_{DC} = \frac{3\sqrt{2}}{\pi} U_{eff} = 1.35 U_{eff} = 1.35 \cdot 400 = 540 \mathrm{V}
\end{equation}

Weiter entsteht ein Spannungsabfall von 2V über den Leistungshalbleiter vom Stromrichter. Falls die Netzspannung gewisse Schwankungen aufweist darf mit einer Zwischenkreisspannung von $U_{DC} = 530 \mathrm{V}$ gerechnet werden. 

Aus einem gesteuertem Stromrichter ist, gemäss dem Skript von LEA2 (Seite 120), die maximale Spannung von Gleichung \ref{equ:Max_Ausgangsspannung} zu erwarten. Es handelt sich dabei um den Effektivwert der Spannung für einen Strang der Maschine (Sternspannung).

\begin{equation}\label{equ:Max_Ausgangsspannung}
U_{Strang,max} = \frac{U_{DC}}{2} \cdot 1.15 \cdot \frac{1}{\sqrt{2}} = \frac{530}{2} \cdot 1.15 \cdot \frac{1}{\sqrt{2}} = 215.5 \mathrm{V}
\end{equation}

Um mit der Regelung noch einen gewissen Spielraum zu haben, wird eine Reserve von $20\%$ eingerechnet. Die nutzbare Strangspannung liegt dementsprechend bei $U_{Strang,neu} = U_{Strang,max} \cdot 0.8 = 172.4 \mathrm{V}$ 
und die Dreieckspannung bei $U_{N,neu} = U_{Strang,neu} \cdot \sqrt{3} = 298.6 \mathrm{V}$

Mit einer tieferen Spannung über den Maschine folgt eine neue Nenndrehzahl bevor der Feldschwächungsbereich beginnt. Aus der Gleichung \ref{equ:Nennmoment} ist diese Drehzahl zu entnehmen.

\begin{equation}\label{equ:Nenndrehzahl_neu}
\omega _{me,neu} = \frac{U_{Strang,neu} \cdot \sqrt{2}}{p \cdot \Psi_{R,N}} = \frac{172.4 \cdot \sqrt{2}}{2 \cdot 0.896} = 136.1 \mathrm{rad/s}
\end{equation}

Auch die Leistung der Maschine nimmt durch eine tiefer Speisespannung ab. Prinzipiell kann aber immer noch der Selbe Strom fliessen. Es folgt für die Leitung an der Welle:

\begin{equation}\label{equ:Leistung_Welle_neu}
P _{me,neu} = \sqrt{3}\cdot U_{Strang,neu}\cdot I_{N}\cdot \cos \varphi \cdot \eta = \sqrt{3}\cdot 298.6\cdot 11.3\cdot 0.8 \cdot 0.878 = 4.106 \mathrm{kW}
\end{equation}

Mit dieser neuen Leitung kann das neue Drehmoment bzw. das neue Lastmoment bestimmt werden. Es berechnet sich wie folgt:

\begin{equation}\label{equ:Lastmoment_neu}
M_{N,neu} = \frac{P_{me,neu}}{\omega _{me,neu}} = \frac{4.106\cdot 10^{3}}{136.1} = 30.18 \mathrm{Nm}
\end{equation}

Um alle systembedingte Änderungen auf einen Blick zu haben dient die Tabelle \ref{tab:Vergleich_Grenzen}.

\renewcommand{\arraystretch}{1.5}
\begin{table}[h]
	\centering
	\begin{tabular}{|l|c|c|}
		\hline
		\textbf{Parameter} & \textbf{Original} & \textbf{Neu} \\ \hline
		Nennspannung $U_{N}$  & $400\mathrm{V}$ & $298.6\mathrm{V}$ \\ \hline
		Nennstrom $I_{N}$ & $11.3\mathrm{A} $ & $11.3\mathrm{A} $ \\ \hline 
		Nennleistung $P_{me}$ & $5.5\mathrm{kW}$ & $4.106\mathrm{kW}$\\ \hline 
		cos$\varphi$ & 0.8 & 0.8 \\ \hline
		Nenndrehzahl $\omega_{me}$ & 1465U/min & 1300U/min \\ \hline
		Nennmoment $M_{N}$ & $35.86\mathrm{Nm}$ & $30.18\mathrm{Nm}$ \\ \hline
		Zwischenkreisspannung $U_{DC}$ & - & 530V \\ \hline
	\end{tabular} 	
	\caption{Vergleich zwischen den Originalgrössen und den neuen Grössen bestimmt durch die Zwischenkreisspannung}
	\label{tab:Vergleich_Grenzen}
\end{table}
 
\renewcommand{\arraystretch}{1}
 



