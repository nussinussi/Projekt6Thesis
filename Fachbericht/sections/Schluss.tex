\section{Schluss}

\subsection{Reflektion} \todo{komplett neu erstellen}

In diesem Kapitel werden die Teilziele vom Pflichtenheft Reflektiert. In der untenstehenden Tabelle \ref{tab: Teilziele} sind die zu beginn des Projekts Definierten Teilziele so wie der Status, in welchem der Fachbericht am ende verfasst wurde.
\begin{table}[h!]
	\begin{tabular}{|c|l|c|}
		\hline 
		Teilziel & Definition & Status \\ 
		\hline 
		1 & Definitive vereinbarung ist unterschrieben & Erfüllt \\ 
		\hline 
		2 & Analyse und Evaluation Mikrocontroller & Erfüllt \\ 
		\hline 
		3 & Evaluation Funk Verbindung & Erfüllt \\ 
		\hline 
		4 & Evaluation Hardware Komponenten & Erfüllt \\ 
		\hline 
		5 & Analyse von verschiedenen FW-Frameworks & Erfüllt \\ 
		\hline 
		6 & Konzept der HW Infrastruktur erstellt & Erfüllt \\ 
		\hline 
		7 & Funktionsmuster, Fliegender Aufbau, Hotspotfunktion für& \\
		& Grundkonfiguration, MQTT Nachrichten können versendet werden & Erfüllt \\ 
		\hline 
		8 & Evaluation Server Software & Erfüllt \\ 
		\hline 
		9 & Dokumentation abgeschlossen & Erfüllt \\ 
		\hline 
		\end{tabular}
	\caption{Teilziele}
	\label{tab: Teilziele}	 
\end{table}



Die Teilziele wurden erreicht, wobei im verlauf des Projekts die Erkenntnis gewonnen wurde, dass die Evaluationen von Frameworks Teilziel 5 und Evaluation Server Software Teilziel 8, sehr viel Zeit in Beanspruchten. Im Idealfall hätte ein komplettes Funktionsmuster pro Framework aufgebaut werden sollen, wozu aber nicht genügen Zeit vorhanden war. Es konnte nur gerade ein Framework ausführlich getestet werden. Die Evaluation Server Software wurde auf zwei Favoriten beschränkt. Die Meilensteine konnten erreicht werden, wobei die der Termin Plan vom Projekt zeitlich nicht eingehalten wurde, so mussten durch lösen von verschiedenen Problemen konnte in er Projektwoche nicht alle vorgesehen arbeiteten erledigt werden und so kamen alle nachstehenden Arbeiten in Verzug.

\subsection{Ausblick}
Damit nicht die gleichen Probleme wie im Vergangenen Projekt auftauchen, wird ein Fixer Tag für das Projekt vorgesehen, Voraussichtlich ist dies der Samstag. Weil so viel Zeit zum Lösen von Problemen drauf gegangen war, wurde vorgenommen vermehrt die Hilfe vom Betreuer beanspruchen, somit sollen mehr Sitzungen stattfinden als im Projekt 5.

