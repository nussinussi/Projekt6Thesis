\section{Schluss}

\subsection{Reflektion}

In diesem Kapitel werden die Projekt Ziele Reflektiert. In der untenstehenden Tabelle \ref{tab: Teilziele} sind die zu beginn des Projekts Definierten Teilziele so wie der Status, in welchem der Fachbericht am ende verfasst wurde.
\begin{table}[h!]
	\small
	\begin{tabular}{|c|l|c|}
		\hline 
		Definition & Status \\ 
		\hline 
		Sensor- und Aktorbaustein sind als Prototypen hergestellt& Erfüllt \\ 
		\hline 
		Analyse und Evaluation Mikrocontroller & Erfüllt \\ 
		\hline 
		Frameware ist kompilierbar und erlaubt eine Inbetriebnahme der Prototypen & Erfüllt \\ 
		\hline 
		Signalverknüpfungen sind mit OpenHab realisierbar & Erfüllt \\ 
		\hline 
		Sensor- und Aktorbaustein sind optimiert (Redesign), Mechanisches Gehäuse ist erstellt  & Erfüllt \\ 
		\hline 
		Systemeinrichtung gestaltet sich Benutzerfreundlich & Erfüllt \\ 
		\hline 
		Validierung des gesamten Systems, Testergebnisse sind Dokumentiert n & Erfüllt \\ 
		\hline 
		Befehle können mit Sprachsteuerung ausgeführt werden & Erfüllt \\ 
		\hline 
		Dokumentation abgeschlossen & Erfüllt \\ 
		\hline 
		\end{tabular}
	\caption{Teilziele}
	\label{tab: Teilziele}	 
\end{table}



\subsubsection{Fazit}
 In diesem Projekt wurden Erfahrungen gewonnen, wie eine Hardware Lösung für eine IOT-Anwendung aussehen kann. Es gab einige Probleme wegen des verwendeten Mikrocontrollers, insbesondere des ADCs. Es wurde darauf verzichtet einen externen ADC zu verwenden und andere Lösungen gesucht, sowie die möglichen Verbesserungen der momentanen Lösung erarbeitet. Es hat sich darüber als schwierig herausgestellt, die Temperatur exakt zu messen, da die Umgebungsbedingungen anders sein können, wie stehende Luft und erwärmende Frontplatte. Weitere Erkenntnisse wurden ebenfalls im Bereich des Google Assistent gewonnen, durch die dynamische DNS Adresse  und SSL-Zertifikate gestaltete sich die Integration nicht mit einer direkten Verbindung, siehe Kapitel Sprachassistent.

