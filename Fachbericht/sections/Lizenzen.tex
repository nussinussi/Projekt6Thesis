\section{Lizenzen}
In diesem Kapitel werden die Lizenzen, welche in der Software genutzt werden, behandelt. In der Tabelle \ref{tab: Lizensen} sind die Lizenzen der verwendeten Bibliothek aufgeführt und für was sie verwendet wurden. Die Standard GNU GPL erlaubt die Verwendung nur für freie Software und die aufgeführte GNU LGPL auch für proprietäre Software \cite{noauthor_gnuorg_nodate}. Iostream ist unter GNU Lizenz, was unsere Software, ebenso zur freien Software macht. Der Grund hierfür ist die strenge Copyleft-Klausel der GNU GPL. Diese sagt aus, dass eine veränderte Fassung des ursprünglichen Werkes, nicht mehr Nutzungseinschränkungen haben darf, als das Original.\cite{noauthor_copyleft_2020} Die MIT lizenzierte Software kann  unter anderen Lizenzen gestellt werden \cite{noauthor_mit_2020}. Die Apache Lizenz erlaubt es die Software für jeglichen Grund zu verwenden und dies ohne Lizenzgebühren \cite{noauthor_apache_2020}. Zusammenfassend ist zu sagen, dass unsere Software, als freie Software angeboten werden müsste.

\begin{table}[]
	\small
	\begin{tabular}{|c|c|c|c|}
		\hline
		\textbf{Bibliothek} & \textbf{Lizenz} & \textbf{Herausgeber} & \textbf{Zweck} \\ \hline
		Arduino.h & \begin{tabular}[c]{@{}c@{}}GNU LGPL  2.1 \\ oder höher\end{tabular} & Arduino & \begin{tabular}[c]{@{}c@{}}Für Arduino Funktionen \\ notwendig\end{tabular} \\ \hline
		Fs.h & \begin{tabular}[c]{@{}c@{}}GNU LGPL  2.1 \\ oder höher\end{tabular} & Ivan Grokhotkov & \begin{tabular}[c]{@{}c@{}}Ermöglicht Filesystem \\ für den Web-Server\end{tabular} \\ \hline
		SPIFFS.h & Apache License 2.0 & Espressif Systems & \begin{tabular}[c]{@{}c@{}}Notwendig für das\\  Filesystem des ESP32\end{tabular} \\ \hline
		iostream & \begin{tabular}[c]{@{}c@{}}GNU GPL  3.0\\  oder höher\end{tabular} & \begin{tabular}[c]{@{}c@{}}Free Software \\ Foundation\end{tabular} & \begin{tabular}[c]{@{}c@{}}Braucht es um Daten \\ zu Buffern mit C++\end{tabular} \\ \hline
		WiFi.h & \begin{tabular}[c]{@{}c@{}}GNU LGPL  2.1 \\ oder höher\end{tabular} & Arduino & \begin{tabular}[c]{@{}c@{}}Braucht es um ein Wifi \\ einzurichten und\\  ermöglicht SSID und WPA2\end{tabular} \\ \hline
		esp\_wifi.h & Apache License 2.0 & Espressif Systems & \begin{tabular}[c]{@{}c@{}}WIrd benötigt,  damit das Wifi\\  unter dem ESP32 läuft\end{tabular} \\ \hline
		WebServer.h & \begin{tabular}[c]{@{}c@{}}GNU LGPL  2.1\\  oder höher\end{tabular} & Ivan Grokhotkov & Erstellt den Web-Server \\ \hline
		DNSServer.h & \begin{tabular}[c]{@{}c@{}}Verwendet Bibliothek \\ mit MIT Lizenz\end{tabular} & Bjoern Hartmann & \begin{tabular}[c]{@{}c@{}}Port für den DNS-Server \\ des ESP32\end{tabular} \\ \hline
		WiFiManager.h & MIT license & github: tzapu & \begin{tabular}[c]{@{}c@{}}Konfigurieren der \\ WiFi-Zugangsdaten\end{tabular} \\ \hline
		esp\_system.h & Apache License 2.0 & Espressif Systems & \begin{tabular}[c]{@{}c@{}}Konfigueriert die\\  Perepherie des ESP32\end{tabular} \\ \hline
		stdio.h & \begin{tabular}[c]{@{}c@{}}Nur Kopieren des \\ Copyright Absatzes\end{tabular} & \begin{tabular}[c]{@{}c@{}}University of \\ California\end{tabular} & \begin{tabular}[c]{@{}c@{}}Beinhaltet grundlegende\\  Funktionen, Makros usw.\end{tabular} \\ \hline
		time.h & GNU C & - & \begin{tabular}[c]{@{}c@{}}Makros und Funktionen um \\ mit der Zeit zu rechnen\end{tabular} \\ \hline
		PubSubClient.h & \begin{tabular}[c]{@{}c@{}}GNU LGPL  2.1\\  oder höher\end{tabular} & Nick O'Leary & Erstellt einen MQTT Client \\ \hline
		driver/adc.h & Apache License 2.0 & Espressif Systems & ADC Teiber \\ \hline
		ArduinoJson.h & MIT Lizens & Benoit Blanchon & \begin{tabular}[c]{@{}c@{}}allgemeine library für C++, \\ wird für den Buffer verwendet\end{tabular} \\ \hline
	\end{tabular}
	\caption{Verwendete Bibliothen und deren Lizensen}
	\label{tab: Lizensen}
\end{table}