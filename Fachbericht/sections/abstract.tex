\begin{abstract}

Smart Home is the automation, and data-based control of electronic devices and light energy systems in your own home. This report is about a smart home system solution with Mqtt communication. The development of an actor board with different switching possibilities, where electronic devices are connected, as well as the development of a touch sensor board with temperature measurement are described. A closed system with its own MQTT-broker is completed with the server software.The server software enables a perfect interaction between the components and the integration of a language assistant. The whole system can be integrated with existing Smart Home and offers cross-platform compatibility with other solutions.


%Der Softwareteil befasst sich mit verschiedenen Komponenten, so wird ein Framework für den Sensor/Aktorbaustein wie auch das Framework für den Inhouse Server und das MQTT Konzept beschrieben. Das Hardwarekonzept des Sensorbausteins besteht im wesentlichen aus einem WiFi fähigen Mikrocontroller, einem Temperatursensor, Buttons und LEDs. Dabei besteht der Sensorbaustein aus drei physisch getrennten Teilen: der Spannungsversorgung, dem Hauptprint und der Frontprint. Der Frontprint beinhaltet Touchsensoren und LEDs, welche einen individualisierten Einschaltzustand, z.B Lüftung ist an, signalisieren. Das wichtigste im Aktorbaustein ist ebenso ein Mikrocontroller mit WiFi, dazu kommen noch Relais, $10\;V$ Ein-/Ausgänge und LEDs, um den Schaltzustand der Relais zu signalisieren.

%Das Abstract ist eine Art Zusammenfassung des ganzen Dokuments. Es gibt einen Einblick in die Aufgabenstellung, wie diese umgesetzt wurde und welches Ergebnis erreicht wurde. Aus diesem Grund wird das Abstract immer ganz am Schluss der Arbeit verfasst. Es besteht aus einem zusammengehörenden Absatz und umfasst ungefähr 10 bis 20 Zeilen.Formeln, Referenzen oder andere Unterbrechungen haben im Text nichts zu suchen.Direkt unter dem Abstract folgt eine Liste von drei bis vier Stichworten/Keywords. Diese werden in alphabetischer Reihenfolge aufgelistet und beschreiben das Themengebiet der Arbeit.

%\vspace{2ex}
%\textbf{Keywords: Anleitung, LaTeX, Thesis, Vorlage}
\end{abstract}	